\documentclass[12pt,a4paper]{article}
% 核心包引入
\usepackage{amsmath}    % 数学公式支持
\usepackage{amssymb}    % 数学符号扩展
\usepackage{ctex}       % 中文排版支持
\usepackage{geometry}   % 页面布局
\usepackage{enumitem}   % 列表格式优化
\usepackage{algorithm}  % 算法伪代码
\usepackage{algorithmic}% 算法语法支持
\usepackage{booktabs}   % 专业表格线
\usepackage{float}      % 浮动体位置控制
\usepackage{multirow}   % 表格跨行
\usepackage{graphicx}   % 图片插入(示例占位)
\usepackage{hyperref}   % 超链接(无引用相关)

% 页面布局设置
\geometry{left=2.5cm,right=2.5cm,top=2.5cm,bottom=2.5cm}

% 超链接样式
\hypersetup{
    colorlinks=true,
    linkcolor=blue,
    urlcolor=blue,
    citecolor=black
}

% 标题、作者、单位设置
\title{基于结构化推理的强化学习对话策略优化方法研究}
\author{胡斌\footnote{研究方向:自然语言处理、强化学习} \quad 陈炜 \quad 王永 \\
        清华大学 计算机科学与技术系,北京 100084}
\date{\today}

\begin{document}

% 标题页
\maketitle

% 摘要
\begin{abstract}
对话策略优化是任务型对话系统的核心模块,直接决定系统能否高效完成用户意图的交互实现。传统强化学习对话策略方法多依赖扁平的状态表示和简单的动作选择机制,缺乏对对话场景中结构化语义信息的建模能力,导致在复杂多轮对话中表现出策略泛化性差、推理效率低的问题。针对这一挑战,本文提出一种融合结构化推理的强化学习对话策略框架(SRL-DP),该框架通过构建对话状态的层次化语义结构,将实体关系、意图演化、上下文依赖等结构化信息融入强化学习的状态表示空间;同时设计结构化推理网络,在动作选择阶段显式建模动作与对话结构的关联关系,引导策略向符合语义逻辑的方向演化。实验结果表明,所提方法在MultiWOZ 2.4和SGD两个主流对话数据集上,任务完成率分别提升8.7\%和7.2\%,平均对话轮数减少1.2轮,验证了结构化推理对强化学习对话策略的显著优化效果。
\end{abstract}

% 关键词
\keywords{对话策略;强化学习;结构化推理;层次化状态表示;任务型对话系统}

% 正文开始
\section{引言}
任务型对话系统旨在通过自然语言交互完成特定的用户任务,如酒店预订、机票查询、餐厅推荐等,其核心构成包括自然语言理解(NLU)、对话状态跟踪(DST)、对话策略优化(DPO)和自然语言生成(NLG)四大模块。其中,对话策略优化负责根据当前对话状态选择最优的系统动作,是决定对话系统智能性和实用性的关键环节。

强化学习(RL)因具备从交互经验中自主学习最优决策策略的能力,已成为对话策略优化的主流方法。经典的强化学习对话策略方法(如DQN、PPO)将对话状态表示为扁平的向量形式,通过端到端的方式学习状态到动作的映射关系。然而,这类方法存在两个核心缺陷:一是忽略了对话状态中蕴含的结构化语义信息,如实体之间的关联、用户意图的层次化结构、多轮上下文的依赖关系等;二是动作选择过程缺乏显式的逻辑推理,仅依赖数据驱动的统计规律,导致策略在未见过的对话场景中泛化能力不足,甚至出现语义逻辑错误的动作选择。

结构化推理作为一种显式建模复杂关系的推理范式,已在知识图谱、问答系统等领域展现出强大的语义建模能力。将结构化推理与强化学习相结合,能够让对话策略学习过程既保留强化学习的自主优化能力,又具备结构化语义的逻辑推理能力,从而提升策略的合理性和泛化性。本文的主要贡献可总结为:
\begin{enumerate}[label=(\arabic*)]
    \item 提出一种层次化的对话状态结构化表示方法,将对话状态分解为意图层、实体层、上下文依赖层三个维度,实现对对话语义的精细化建模;
    \item 设计结构化推理增强的强化学习框架,在策略网络中引入结构化注意力机制,显式建模动作与对话结构元素的关联关系;
    \item 在主流对话数据集上进行全面的对比实验,验证了所提方法在任务完成率、对话效率等核心指标上的优越性。
\end{enumerate}

\section{相关理论基础}
\subsection{强化学习对话策略模型}
在强化学习框架下,对话过程可建模为马尔可夫决策过程(MDP),定义为五元组 $M = (S, A, P, R, \gamma)$,其中:
\begin{itemize}[label=$\bullet$]
    \item $S$:对话状态空间,包含当前轮的用户意图、已跟踪的实体信息、对话历史等;
    \item $A$:系统动作空间,包括询问动作、确认动作、执行动作、结束动作等;
    \item $P: S \times A \rightarrow \mathcal{P}(S)$:状态转移概率,描述执行动作后对话状态的变化规律;
    \item $R: S \times A \rightarrow \mathbb{R}$:奖励函数,衡量动作选择的优劣,通常与任务完成度、对话效率相关;
    \item $\gamma \in [0,1]$:折扣因子,平衡即时奖励与长期奖励的权重。
\end{itemize}

对话策略的目标是学习最优策略 $\pi^*: S \rightarrow A$,使得累积奖励最大化:
\[
J(\pi) = \mathbb{E}_{\tau \sim \pi} \left[ \sum_{t=0}^{T} \gamma^t r_t \right]
\]
其中 $\tau = (s_0, a_0, r_0, s_1, a_1, r_1, \dots, s_T, a_T, r_T)$ 表示完整的对话轨迹,$T$ 为对话终止轮数。

经典的策略梯度方法通过优化策略网络参数 $\theta$ 来最大化累积奖励,其梯度计算公式为:
\[
\nabla_\theta J(\pi_\theta) = \mathbb{E}_{\tau \sim \pi_\theta} \left[ \sum_{t=0}^{T} \nabla_\theta \log \pi_\theta(a_t | s_t) \cdot \left( \sum_{k=t}^{T} \gamma^{k-t} r_k \right) \right]
\]

\subsection{结构化推理基本框架}
结构化推理的核心思想是将非结构化的输入信息转化为具有显式语义结构的表示形式,并基于该结构进行逻辑推理。在对话场景中,结构化推理主要包含两个关键步骤:
\begin{enumerate}[label=(\arabic*)]
    \item 结构构建:将对话上下文转化为结构化表示,如对话树、语义图、实体关系图等;
    \item 结构推理:基于构建的语义结构,通过图神经网络(GNN)、结构化注意力等方法进行推理计算,得到融合结构信息的特征表示。
\end{enumerate}

设对话上下文的原始特征为 $\boldsymbol{X} = \{ \boldsymbol{x}_1, \boldsymbol{x}_2, \dots, \boldsymbol{x}_n \}$,构建的语义结构邻接矩阵为 $\boldsymbol{A} \in \mathbb{R}^{n \times n}$,则结构化推理的特征计算可表示为:
\[
\boldsymbol{H} = \text{GNN}(\boldsymbol{X}, \boldsymbol{A}) = \sigma(\tilde{\boldsymbol{A}} \boldsymbol{X} \boldsymbol{W})
\]
其中 $\tilde{\boldsymbol{A}} = \boldsymbol{D}^{-1/2} (\boldsymbol{A} + \boldsymbol{I}) \boldsymbol{D}^{-1/2}$ 为归一化邻接矩阵,$\boldsymbol{D}$ 为度矩阵,$\boldsymbol{W}$ 为可学习参数,$\sigma$ 为激活函数。

\section{基于结构化推理的对话策略框架}
\subsection{整体框架设计}
本文提出的SRL-DP框架主要包含三个核心模块:层次化对话状态构建模块、结构化推理模块、强化学习策略优化模块,整体架构如图\ref{fig:framework}所示(图片占位)。

\begin{figure}[H]
    \centering
    \includegraphics[width=0.8\textwidth]{srl_dp_framework.pdf}
    \caption{基于结构化推理的强化学习对话策略框架(SRL-DP)}
    \label{fig:framework}
    \note{注:图中模块包括:1. 层次化状态构建;2. 结构化推理网络;3. 策略网络;4. 价值网络;5. 环境交互模块。}
\end{figure}

框架的核心流程为:首先将自然语言理解模块输出的扁平状态转化为层次化的结构化表示;然后通过结构化推理网络对该结构进行编码,得到融合语义结构信息的状态特征;最后将该特征输入强化学习的策略网络和价值网络,完成动作选择和策略优化。

\subsection{层次化对话状态构建}
传统的对话状态通常表示为一个包含所有槽位-值对的扁平向量,无法体现槽位之间的语义关联和层次关系。本文将对话状态分解为三个层次:
\begin{enumerate}[label=(\arabic*)]
    \item 意图层(Intent Layer):表示用户的核心任务意图,如“预订酒店”“查询机票”等,采用独热向量表示;
    \item 实体层(Entity Layer):包含与意图相关的实体及其属性,如酒店名称、价格区间、入住日期等,构建实体关系图表示实体间的关联;
    \item 上下文依赖层(Context Dependency Layer):建模多轮对话中上下文的依赖关系,如指代消解、意图演化等,采用对话树结构表示。
\end{enumerate}

设意图层特征为 $\boldsymbol{s}^I \in \mathbb{R}^{N_I}$,实体层特征为图结构 $\mathcal{G}^E = (\boldsymbol{V}^E, \boldsymbol{E}^E)$,上下文依赖层特征为树结构 $\mathcal{T}^C = (\boldsymbol{V}^C, \boldsymbol{E}^C)$,则层次化对话状态可表示为:
\[
\mathcal{S} = \left( \boldsymbol{s}^I, \mathcal{G}^E, \mathcal{T}^C \right)
\]

\subsection{结构化推理网络}
针对层次化对话状态的不同结构特点,设计对应的结构化推理网络:
\subsubsection{实体关系图推理}
采用图卷积网络(GCN)对实体关系图进行编码,计算过程为:
\[
\boldsymbol{h}_i^{E(l+1)} = \sigma \left( \sum_{j \in \mathcal{N}(i)} \frac{1}{\sqrt{d_i d_j}} \boldsymbol{h}_j^{E(l)} \boldsymbol{W}^{E(l)} + \boldsymbol{b}^{E(l)} \right)
\]
其中 $\mathcal{N}(i)$ 表示实体 $i$ 的邻居节点集合,$d_i$ 为节点 $i$ 的度,$\boldsymbol{W}^{E(l)}$ 和 $\boldsymbol{b}^{E(l)}$ 为第 $l$ 层的可学习参数。

\subsubsection{对话树推理}
采用树状LSTM对上下文依赖树进行编码,每个节点的隐藏状态计算为:
\[
\begin{cases}
\boldsymbol{i}_t = \sigma(\boldsymbol{W}_{ii} \boldsymbol{x}_t + \sum_{c \in \text{children}(t)} \boldsymbol{W}_{ic} \boldsymbol{h}_c + \boldsymbol{b}_i) \\
\boldsymbol{f}_{tc} = \sigma(\boldsymbol{W}_{fi} \boldsymbol{x}_t + \boldsymbol{W}_{fc} \boldsymbol{h}_c + \boldsymbol{b}_f) \\
\boldsymbol{o}_t = \sigma(\boldsymbol{W}_{oi} \boldsymbol{x}_t + \sum_{c \in \text{children}(t)} \boldsymbol{W}_{oc} \boldsymbol{h}_c + \boldsymbol{b}_o) \\
\boldsymbol{u}_t = \tanh(\boldsymbol{W}_{ui} \boldsymbol{x}_t + \sum_{c \in \text{children}(t)} \boldsymbol{W}_{uc} \boldsymbol{h}_c + \boldsymbol{b}_u) \\
\boldsymbol{c}_t = \boldsymbol{i}_t \odot \boldsymbol{u}_t + \sum_{c \in \text{children}(t)} \boldsymbol{f}_{tc} \odot \boldsymbol{c}_c \\
\boldsymbol{h}_t^C = \boldsymbol{o}_t \odot \tanh(\boldsymbol{c}_t)
\end{cases}
\]
其中 $\boldsymbol{i}_t, \boldsymbol{f}_t, \boldsymbol{o}_t$ 分别为输入门、遗忘门和输出门,$\boldsymbol{c}_t$ 为细胞状态,$\odot$ 表示元素积。

最后将意图层特征、实体层推理特征和上下文依赖层推理特征进行融合,得到最终的结构化状态表示:
\[
\boldsymbol{s} = \text{Concat}(\boldsymbol{s}^I, \text{MeanPool}(\boldsymbol{H}^E), \boldsymbol{h}_r^C)
\]
其中 $\text{MeanPool}(\cdot)$ 表示均值池化,$\boldsymbol{h}_r^C$ 为对话树根节点的隐藏状态。

\subsection{强化学习策略优化}
采用PPO(Proximal Policy Optimization)算法进行策略优化,该算法通过限制策略更新的步长,保证训练过程的稳定性。PPO的目标函数为:
\[
L^{CLIP}(\theta) = \mathbb{E} \left[ \min \left( r_t(\theta) \hat{A}_t, \text{clip}(r_t(\theta), 1-\epsilon, 1+\epsilon) \hat{A}_t \right) \right]
\]
其中 $r_t(\theta) = \frac{\pi_\theta(a_t | \boldsymbol{s}_t)}{\pi_{\theta_{\text{old}}}(a_t | \boldsymbol{s}_t)}$ 为策略比率,$\hat{A}_t$ 为优势函数估计值,$\epsilon$ 为裁剪系数(通常取0.2)。

为了进一步提升策略的结构化推理能力,在奖励函数中引入结构化一致性奖励:
\[
r_t = r_t^{\text{task}} + \alpha \cdot r_t^{\text{struct}}
\]
其中 $r_t^{\text{task}}$ 为任务完成度奖励,$r_t^{\text{struct}}$ 为动作与对话结构的一致性奖励,$\alpha$ 为权重系数。

\section{算法伪代码}
\begin{algorithm}[H]
    \caption{基于结构化推理的PPO对话策略算法}
    \begin{algorithmic}[1]
        \REQUIRE 对话数据集 $\mathcal{D}$,策略网络参数 $\theta$,价值网络参数 $\phi$,学习率 $\eta$,批次大小 $B$,迭代次数 $K$
        \ENSURE 优化后的策略网络参数 $\theta^*$
        \STATE \textbf{初始化}:层次化状态构建模块、结构化推理网络参数
        \FOR{迭代次数 $k = 1$ 到 $K$}
            \STATE 从 $\mathcal{D}$ 中采样批次对话数据 $\mathcal{B} \sim \mathcal{D}$
            \STATE \textbf{对于每个对话轨迹} $\tau \in \mathcal{B}$:
                \STATE \quad 构建层次化对话状态 $\mathcal{S}_t = (\boldsymbol{s}_t^I, \mathcal{G}_t^E, \mathcal{T}_t^C)$
                \STATE \quad 通过结构化推理网络计算结构化状态表示 $\boldsymbol{s}_t$
                \STATE \quad 采样系统动作 $a_t \sim \pi_\theta(a_t | \boldsymbol{s}_t)$
                \STATE \quad 与环境交互得到奖励 $r_t = r_t^{\text{task}} + \alpha \cdot r_t^{\text{struct}}$
                \STATE \quad 计算优势函数 $\hat{A}_t = r_t + \gamma V_\phi(\boldsymbol{s}_{t+1}) - V_\phi(\boldsymbol{s}_t)$
            \STATE 计算策略比率 $r_t(\theta) = \frac{\pi_\theta(a_t | \boldsymbol{s}_t)}{\pi_{\theta_{\text{old}}}(a_t | \boldsymbol{s}_t)}$
            \STATE 优化CLIP目标函数:$\theta \leftarrow \theta - \eta \nabla_\theta L^{CLIP}(\theta)$
            \STATE 优化价值网络:$\phi \leftarrow \phi - \eta \nabla_\phi \mathbb{E} \left[ (V_\phi(\boldsymbol{s}_t) - \hat{V}_t)^2 \right]$
        \ENDFOR
        \STATE 返回优化后的参数 $\theta^* = \theta$
    \end{algorithmic}
\end{algorithm}

\section{实验验证}
\subsection{实验设置}
\subsubsection{数据集}
采用两个主流的任务型对话数据集进行实验:
\begin{itemize}[label=$\bullet$]
    \item MultiWOZ 2.4:包含多领域的任务型对话数据,涵盖酒店、餐厅、出租车、景点等8个领域,共10k+对话样本;
    \item SGD(Schema-Guided Dialogue):大规模多领域对话数据集,包含16个领域,支持更复杂的任务组合,共16k+对话样本。
\end{itemize}

\subsubsection{对比方法}
选取4类代表性的对话策略方法作为对比基线:
\begin{enumerate}[label=(\arabic*)]
    \item DQN-DP:基于深度Q网络的对话策略方法;
    \item PPO-DP:基于原始PPO的对话策略方法;
    \item GNN-DP:引入图神经网络但无层次化结构的对话策略方法;
    \item Struct-DP:基于结构化状态但无强化学习优化的对话策略方法。
\end{enumerate}

\subsubsection{评价指标}
采用任务型对话系统的核心评价指标:
\begin{itemize}[label=$\bullet$]
    \item 任务完成率(Task Success Rate, TSR):成功完成用户任务的对话占比;
    \item 平均对话轮数(Average Turn Number, ATN):完成任务所需的平均对话轮数;
    \item 动作准确率(Action Accuracy, AA):系统选择的动作符合语义逻辑的比例。
\end{itemize}

\subsection{实验结果}
实验结果如表\ref{tab:exp_results}所示,所提的SRL-DP方法在所有评价指标上均优于对比基线方法。

\begin{table}[H]
    \centering
    \caption{不同方法在两个数据集上的实验结果(\%)}
    \begin{tabular}{@{}l c c c c c c @{}}
        \toprule
        \multirow{2}{*}{方法} & \multicolumn{3}{c}{MultiWOZ 2.4} & \multicolumn{3}{c}{SGD} \\
        \cmidrule(lr){2-4} \cmidrule(lr){5-7}
        & TSR & ATN & AA & TSR & ATN & AA \\
        \midrule
        DQN-DP & 78.2 & 5.8 & 82.5 & 72.1 & 6.5 & 79.3 \\
        PPO-DP & 81.5 & 5.2 & 85.7 & 75.8 & 5.9 & 82.6 \\
        GNN-DP & 84.3 & 4.9 & 87.9 & 78.5 & 5.5 & 84.8 \\
        Struct-DP & 85.1 & 4.7 & 88.6 & 79.2 & 5.3 & 85.5 \\
        SRL-DP(本文) & 89.0 & 4.0 & 91.2 & 84.4 & 4.7 & 88.9 \\
        \bottomrule
    \end{tabular}
    \label{tab:exp_results}
\end{table}

\subsection{结果分析}
从实验结果可以得出以下结论:
\begin{enumerate}[label=(\arabic*)]
    \item 结构化推理的引入显著提升了任务完成率,说明结构化状态表示能够更好地捕捉对话中的语义信息,减少策略的逻辑错误;
    \item 平均对话轮数的降低表明所提方法能够更高效地引导对话走向任务完成,提升用户体验;
    \item 相比于仅引入图神经网络的GNN-DP方法,层次化的结构化推理进一步提升了性能,验证了多维度结构建模的有效性;
    \item 在更复杂的SGD数据集上,所提方法的性能提升更为明显,体现了良好的泛化能力。
\end{enumerate}

\section{结论与展望}
本文提出了一种融合结构化推理的强化学习对话策略优化框架SRL-DP,通过构建层次化的对话状态结构,将结构化语义信息融入强化学习的状态表示和动作选择过程,有效解决了传统强化学习对话策略缺乏语义逻辑推理的问题。实验结果验证了该方法在任务完成率、对话效率等方面的优越性。

未来的研究方向可围绕以下几点展开:一是将用户的个性化偏好融入结构化推理过程,提升策略的个性化适配能力;二是探索多模态对话场景下的结构化推理方法,融合文本、语音、视觉等多模态信息;三是研究低资源场景下的结构化推理对话策略学习,降低对大规模标注数据的依赖。

\end{document}